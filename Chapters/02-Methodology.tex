\chapter{Methodology}
\label{cp:methodology}
\section{Apparatus}
A flat plate is positioned in the wind tunnel test chamber as seen in \autoref{fig: OpenCircuitWindTunnel}. Downstream of the flat plate is a pressure rake, seen in \autoref{fig: PressureRake}. The data was collected using 3 DSA units for the pressure measurements. The computer configuration of this can be seen in \autoref{fig: DataCollectionTool}.

\begin{figure}[htpb]
    \centering
    \includegraphics[width=0.75\linewidth]{Figures/IMG_0004.jpeg}
    \caption[Picture of the test section]{Picture of the Test Section}
    \label{fig: OpenCircuitWindTunnel}
\end{figure}


\section{Procedures}
\begin{enumerate}
\item Set the wind tunnel speed to 10 m/s
\item Take pressure measurements at 10 streamwise positions specified by the TA %Couldn't remember what the exact measurements were, change as needed
% the measurements were 1-10,15,20,25,30,35,40,45,50,55,60,65 inches
\item After each adjustment, acquire and save the data to a data file
\end{enumerate}

\section{Derivations}

The voltage outputs obtained by computer software can be converted to dynamic pressures, which can then be related to velocities. The local shear stress coefficient is defined in \autoref{eq:shear_stress_coefficient}. The momentum thickness is defined in \autoref{eq:momentum_thickness}. Using \autoref{eq:momentum_thickness} and the data we collected, the momentum thickness can be calculated for each of the distances \citep{lab8-manual}.

\begin{equation} \label{eq:shear_stress_coefficient}
    C_f = \frac{\tau_w}{1/2 \rho U_e^2}
\end{equation}

\begin{equation} \label{eq:momentum_thickness}
    \theta = \int_0^Y\frac{u}{U_e}\left(1 - \frac{u}{U_e}\right)dy
\end{equation}

\noindent  The integral in \autoref{eq:momentum_thickness} was determined using the midpoint Riemann sum between the top and bottom pressure rake taps. The distance between each pressure tap is \num{4}\unit{\text{mm}}. To calculate the local shear stress coefficient, we used its relationship to the momentum thickness in \autoref{eq:shear_to_momentum}. In MATLAB, the \textit{gradient} command is used to estimate $\frac{d\theta}{dx}$.

\begin{equation} \label{eq:shear_to_momentum}
    C_f = 2\frac{d\theta}{dx}
\end{equation}

\noindent However, for the purpose of comparison, \autoref{eq:shear_to_momentum} can be empirically related to \autoref{eq:empirical_shear_stress}.

\begin{equation} \label{eq:empirical_shear_stress}
    C_f = \frac{0.0583}{Re_x^{0.2}}
\end{equation}

\noindent The coefficient of drag is also able to be calculated and related to the momentum thickness. \autoref{eq:coefficient_of_drag} defines the coefficient of drag, where \gls{D} is drag force and \gls{A} is the surface area. The coefficient of drag can be related to momentum thickness, as shown in \autoref{eq:coeff_drag_momentum_thickness}, where \gls{L} is the length of the plate upstream of the measurement point. 


\begin{equation} \label{eq:coefficient_of_drag}
    C_d = \frac{D}{1/2 \rho U_e^2 A}
\end{equation}

\begin{equation} \label{eq:coeff_drag_momentum_thickness}
    C_d = \frac{2\theta}{L}
\end{equation}

\noindent The coefficient of drag can also be estimated, as seen in \autoref{eq:coeff_drag_estimation}.

\begin{equation} \label{eq:coeff_drag_estimation}
    C_d = \frac{.074}{Re_L^{0.2}}
\end{equation}

\noindent The following equations (\autoref{eq:boundary_layer_laminar} and \autoref{eq:boundary_layer_turbulent}) are theoretical estimations of the boundary layer thickness at a distance \gls{x} over a flat plate.

\begin{equation} \label{eq:boundary_layer_laminar}
    \frac{\delta}{x} = \frac{5.0}{\sqrt{\text{Re}_x}}
\end{equation}

\begin{equation} \label{eq:boundary_layer_turbulent}
    \frac{\delta}{x} = \frac{0.37}{\text{Re}_x^\frac{1}{5}}
\end{equation}

\noindent Assuming transition occurs at $Re_x$ = $10^5$, and using  \autoref{eq:boundary_layer_laminar} and \autoref{eq:boundary_layer_turbulent}, we can estimate the thickness of the boundary layer at various locations.
