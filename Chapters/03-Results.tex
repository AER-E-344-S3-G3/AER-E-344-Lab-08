\chapter{Results}
\label{cp:results}

\noindent \autoref{fig:YdelvsUUinfat1in}, \autoref{fig:YdelvsUUinfat9in}, and \autoref{fig:YdelvsUUinfat25in} show the $Y/\delta$ ratio vs. $U/U_{inf}$ ratio at different distances from the leading edge. More plots of $Y/\delta$ vs. $U/U_{inf}$ are located in \autoref{sec:additional_figures}. The 10 locations chosen for analysis are at [1, 3, 5, 7, 9, 15, 25, 35, 45, 55] inches. The x-axis increases as the vertical distance from the flat plate increases.

\begin{figure}[htpb]
    \centering
     \includesvg[width=0.75\linewidth]{Figures/Ydelta vs. UUinf at 1 in.svg}
     \caption
     [Ydelta vs. UUinf at 1 in]
     {Ydelta vs. UUinf at 1 in}
     \label{fig:YdelvsUUinfat1in}
\end{figure}

\begin{figure}[htpb]
    \centering
     \includesvg[width=0.75\linewidth]{Figures/Ydelta vs. UUinf at 9 in.svg}
     \caption
     [Ydelta vs. UUinf at 9 in]
     {Ydelta vs. UUinf at 9 in}
     \label{fig:YdelvsUUinfat9in}
\end{figure}

\begin{figure}[htpb]
    \centering
     \includesvg[width=0.75\linewidth]{Figures/Ydelta vs. UUinf at 25 in.svg}
     \caption
     [Ydelta vs. UUinf at 25 in]
     {Ydelta vs. UUinf at 25 in}
     \label{fig:YdelvsUUinfat25in}
\end{figure}

\noindent \autoref{fig:BoundaryLayerThicknessFromLE} contains the graphs of the theoretical boundary layer at the laminar region and turbulent regions and the calculated thickness from the lab data. The momentum thickness, $\theta$, at the chosen stream-wise distances are also plotted with their values on the right y-axis.

\begin{figure}[htpb]
    \centering
     \includesvg[width=0.75\linewidth]{Figures/Boundary Layer Thickness vs. Distance from LE.svg}
     \caption
     [Boundary Layer Thickness vs Distance from LE]
     {Boundary Layer Thickness vs Distance from LE}
     \label{fig:BoundaryLayerThicknessFromLE}
\end{figure}

\noindent The coefficient of drag from the theoretical calculations and experimental data distances are located in  \autoref{fig:DragCoefficientVsDisToLE}. 

\begin{figure}[htpb]
    \centering
     \includesvg[width=0.75\linewidth]{Figures/Drag Coefficient vs. Distance from LE.svg}
     \caption
     [Drag Coefficient vs Distance from Le]
     {Drag Coefficient vs Distance from Le}
     \label{fig:DragCoefficientVsDisToLE}
\end{figure}

\noindent The coefficient of local shear stress from the theoretical calculations and experimental data distances are located in  \autoref{fig:DragCoefficientVsDisToLE}. 

\begin{figure}[htpb]
    \centering
     \includesvg[width=0.75\linewidth]{Figures/Local Shear Stress Coefficient vs. Distance from LE.svg}
     \caption
     [Local Shear Stress Coefficient vs Distance from Le]
     {Local Shear Stress Coefficient vs Distance from Le}
     \label{fig:LocalShearStressVsDisToLE}
\end{figure}

\noindent Using \autoref{eq:Re} and the subsequent parameters, we find the Reynolds number at the range from 0 to 65 inches in \autoref{fig:ReVsD}. The length values are converted into meters to make the Reynold's Number unit-less. 

\begin{equation}\label{eq:Re}
    Re = \frac{\rho V L}{\mu}
\end{equation}

\begin{enumerate}
    \item[] $\rho = \qty{1.225}{\kilogram\per\meter^3}$
    \item[] $\mu = \qty{18.18e-06}{\pascal\second}$
    \item[] $L = \qtyrange{0}{65}{in}$
    \item[] $V = \qty{11.25}{\meter\per\second}$
\end{enumerate}

\begin{figure}[htpb]
    \centering
     \includesvg[width=0.75\linewidth]{Figures/Reynold's Number vs. Distance from LE.svg}
     \caption
     [Reynold's Number vs. Distance from LE]
     {Reynold's Number vs. Distance from LE.}
     \label{fig:ReVsD}
\end{figure}