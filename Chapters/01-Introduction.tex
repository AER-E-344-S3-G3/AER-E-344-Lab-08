\chapter{Introduction}
\label{cp:introduction}
A pitot rake is an instrument composed of a series of pitot tubes used to measure the velocity profile of a body. In this experiment, the pitot rake is used to track the pressure at various distances between 0 and 70 inches in the flow behind a the leading edge of a flat plate. 

Using pressure measurements from the pitot rake, we will estimate the local shear stress coefficient, \gls{C_f}, and find the coefficient of drag, \gls{C_D}. We will also estimate the boundary layer thickness, \gls{delta}, and compare it the theoretical values. 